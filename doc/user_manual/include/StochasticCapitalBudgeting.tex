\section{Stochastic Capital Budgeting}
\label{sec:StochasticCapitalBudgeting}

The deterministic capital budgeting model previously developed allows for multiple options in how
we select a project. For example, we might select a project via Plan A, Plan B, Plan C, or not
select the project at all. In addition, the deterministic model allows for multiple types of
resources (e.g., capital budgets and O\&M budgets), and further allows for piggybacking constraints.
Our stochastic capital budgeting model illustrates the ideas of prioritization without the
additional features of multiple types of resources, piggybacking, and multiple options for selecting
each project.  The former two features integrate with the prioritization scheme in a straightforward
way, as we will describe below. The latter-most feature proves to have subtle interactions with
the notion of prioritization, and we discuss that in some detail in this section. The model
sketched here is new and, to our knowledge, has not appeared in the literature. Even though the
notation has been sketched above, we develop the full model here so that this section is
self-contained, given that it specifies our “full” mathematical model for stochastic capital
budgeting.

\[
\begin{array}{ll}
%%%%%%%%%%%%%% INDICES AND SET %%%%%%%%%%%%%%%%
\multicolumn{2}{l}{\mbox{\em Indexes and sets:} } \\
t \in T  & \mbox{time periods (years)} \\
i,i^{'}, i^{''} \in I  & \mbox{candidate projects} \\
j, j^{'} \in J_{i}	& \mbox{options for selecting project $i$} \\
i^{'},j^{'} \in IJ_{ij} & \mbox{piggybacking situations} \\
k \in K	& \mbox{types of resources} \\
\omega \in \Omega & \mbox{scenarios}\\
\\
%%%%%%%%%%%%%% DATA %%%%%%%%%%%%%%%%
\multicolumn{2}{l}{\mbox{\em Data:}} \\
a_{ij}^{\omega} & \mbox{reward of selecting project $i$ via option $j$ under scenario $\omega$}  \\
b_{kt}^{\omega} & \mbox{available budget for a resource of type $k$ in year $t$ under scenario $\omega$}\\
c_{ijkt}^{\omega} & \mbox{consumption of resource of type $k$ in year $t$ } \\
& \mbox{if project $i$ is performed via option $j$ under scenario $\omega$}\\
\\
%%%%%%%%%%%%%% DECISION VARS %%%%%%%%%%%%%%%%
\multicolumn{2}{l}{\mbox{\em Decision variables:}}  \\
x_{ij}^{\omega} & \mbox{1 if project $i$ is selected via option $j$ under scenario $\omega$; 0 otherwise} \hspace*{4.0in}\\
s_{ii^{'}} & \mbox{1 if project $i$ has no lower priority than project $i^{'}$; 0 otherwise} \hspace*{4.0in}\\
y_{i}^{\omega} & \mbox{1 if project $i$ is selected for {\it some} option under scenario $\omega$; 0 otherwise} \hspace*{4.0in}\\
z_{ij} & \mbox{1 if project $i$ is selected via option $j$ under {\it some} scenario; 0 otherwise} \hspace*{4.0in}\\
\end{array}
\]

Model formulation:\\

\begin{equation}\label{stoc_obja}
\mathop{\max}_{s,x,y,z} \sum _{ \omega  \in  \Omega }^{}q^{ \omega } \sum _{i \in I}^{} \sum _{j \in J_{i}}^{}a_{ij}^{ \omega }x_{ij}^{ \omega }
\end{equation}

\begin{equation}\label{stoc_objb}
~~~~~~~~~~~~s.t.~~~~~s_{ii^{'}}+s_{i^{'}i} \geq 1,~ i<i^{'}\text{, i, }i^{'} \in I
\end{equation}

\begin{equation}\label{stoc_objc}
~~~~~~~~y_{i}^{ \omega } \geq y_{i^{'}}^{ \omega }+s_{ii^{'}}-1,~ i \neq i^{'}\text{, i, }i^{'} \in I,  \omega  \in  \Omega
\end{equation}

For simplicity, in what follows we will say that variable $s_{ii^{'}}=1$  means that project
$i$  is higher priority than $i^{'}$  even though the variable definition allows for ties,
i.e., the projects being the same priority. Constraint~(\ref{stoc_objb}) indicates that either
project  $i$  is higher priority than project  $i^{'}$  or vice versa, and further allows both
(i.e., a tie). Constraint~(\ref{stoc_objc}) indicates that if project  $i$  is higher priority
than project  $i^{'}$  $s_{ii^{'}}=1$  then if we select the lower priority project
\textit{under some option} then we must also select the higher priority project; if  $s_{ii^{'}}=0$
or if  $y_{i^{'}}^{\omega}=0$  then the constraint is vacuous.\par

\begin{equation}\label{stoc_objd}
 \sum _{i \in I}^{} \sum _{j \in J_{i}}^{}\text{~ c}_{ijkt}^{ \omega }x_{ij}^{ \omega }~  \leq  b_{kt}^{ \omega },~ k \in K, t \in T,  \omega  \in  \Omega
\end{equation}

Constraint~(\ref{stoc_objd}) requires that we be within budget in each time period, for each resource type, and under each scenario.

\begin{equation}\label{stoc_obje}
 \sum _{j \in J_{i}}^{}~~x_{ij}^{ \omega }= y_{i}^{ \omega },~~~~~~~~~~i \in I,~ \omega  \in  \Omega
\end{equation}

\begin{equation}\label{stoc_objf}
y_{i}^{ \omega }=1,~ i \in I,  \omega  \in  \Omega
\end{equation}

Constraint~(\ref{stoc_obje}) defines binary variable  $y_{i}^{ \omega }$  and simultaneously ensures that we select project  $i$  via at most one option. Constraint~(\ref{stoc_objf}) ensures that we select all must-do projects. We note that this illustrates the alternative to the situation in which we must include the {\it Do Nothing}  option among the alternatives for optional projects.\par

\begin{equation}\label{stoc_objg}
x_{i^{'}j^{'}}^{ \omega }  \leq \text{~ x}_{ij}^{ \omega },~~ \left( i^{'},j^{'} \right)  \in IJ_{ij} , j \in J_{i},i \in I
\end{equation}

Constraint~(\ref{stoc_objg}) captures piggybacking conditions.\par

\begin{equation}\label{stoc_objh}
s_{ii^{'}}+s_{i^{'}i} \leq 1,~ i<i^{'}\text{, i, }i^{'} \in I
\end{equation}

\begin{equation}\label{stoc_obji}
s_{ii^{'}~}+s_{i^{'}i^{''}}+s_{i^{''}i} \leq 2,~ i \neq i^{'},i^{'} \neq  i^{''},i^{''} \neq i,~i,~i^{'},i^{''} \in I
\end{equation}

Constraints~(\ref{stoc_objh})-(\ref{stoc_obji}) require that we produce a total ordering of the
projects rather than allowing for ties. If we remove constraints~(\ref{stoc_objh})-(\ref{stoc_obji})
then it will not change the optimal NPV that we obtain, but including the constraints can facilitate
easier parsing of the solutions.\par

\begin{equation}\label{stoc_objj}
x_{i^{'}j}^{ \omega }+s_{ii^{'}~}-1 \leq  \sum _{\begin{array}{c}
	j^{'} \in J_{i}\\
	j^{'} \leq j~\\
	\end{array}}^{}x_{ij}^{ \omega }~,~~i \neq i^{'}~i,~i^{'} \in I,~j \in J_{i^{'}},~ \omega  \in  \Omega
\end{equation}


Constraint~(\ref{stoc_objj}) is a type of consistency constraint with respect to the notion of
options; the constraint matters only when project  $i$  is higher priority than project
$i^{'}$ $s_{ii^{'}}=1$ . In this case, if we select Plan A for the lower priority project
then we must select plan A for the higher priority project. If we select Plan B for the lower
priority project, then we can select Plan A or Plan B for the higher priority project. And,
if we select Plan C for the lower priority project then we can select Plan A, B, or C for the
higher priority project. Inclusion of constraint~(\ref{stoc_objj}) is optional and reflects
how the decision maker prefers to interpret the notion of priorities.

\begin{equation}\label{stoc_objk}
 \sum _{ \omega  \in  \Omega }^{}\text{~ x}_{ij}^{ \omega } \leq   \vert  \Omega  \vert ~z_{ij} ,i \in I, j \in J_{i}
\end{equation}

\begin{equation}\label{stoc_objl}
 \sum _{j \in J_{i}}^{}z_{ij} \leq ~1,~  i \in I
\end{equation}

\begin{equation}\label{stoc_objm}
s_{ii^{'}},x_{ij}^{ \omega },y_{i}^{ \omega },z_{ij} \in { 0,1 } , i \neq i^{'},i,i^{'} \in I, j \in J_{i} ,  \omega  \in  \Omega
\end{equation}
%\begin{comment}

Constraints~(\ref{stoc_objk}) and (\ref{stoc_objl}) taken together indicate that, for each
project separately, we cannot mix use of Plans A, B, and C across different scenarios.
