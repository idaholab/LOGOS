\section{Critical Path Model (CPM)}
\label{sec:CPM}

The CPM model is designed to perform schedule durations calculations provided a set of activities
linked by a graph structure.
This model is designed to be used in a RAVEN workflow where the actitvity duration values can be
changed through a specific sampling strategy (either sampling or optmization).

The schedule requires a ``start'' and an ``end'' activity, a list of additional activities and
their correspoding duration values.
The graph can be imported in two ways. In the first way, the graph is defined in the
\xmlNode{map} nodes. In each instance of the \xmlNode{map} node, an activity is defined and the
following information is required: activity ID (act attribute), activity duration (dur attribute),
and the list of outgoing activities.

Example of CPM input XML in RAVEN input file:
\begin{lstlisting}[style=XML]
  <Models>
    <ExternalModel name="CPMmodel" subType="LOGOS.BaseCPMm odel">
      <variables>start,b,c,d,end,f,g,h,end_time,CP</variables>
      <CPtime>end_time</CPtime>
      <CPid>CP</CPid>
      <map act='start' dur='10'>f,b,h</map>
      <map act='b'     dur='20'>c</map>
      <map act='c'     dur='5' >g,d</map>
      <map act='d'     dur='10'>end</map>
      <map act='f'     dur='15'>g</map>
      <map act='g'     dur='5' >end</map>
      <map act='h'     dur='15'>end</map>
      <map act='end'   dur='20'></map>
    </ExternalModel>
  </Models>
\end{lstlisting}

In the second way, the graph structure can be define in a .py file and it is specified in a
project() class (see below). In this class each activity is define in terms of ID and duration
(see Activity object below).
Then the graph structure is defined through a dictionary, for each activity, a list of list of
outgoing activities is provided.

Example of schedule definition in an external .py file:
\begin{lstlisting}[language=Python]
  from LOGOS.src.CPM.PertMain2 import Pert
  from LOGOS.src.CPM.PertMain2 import Activity

  class project():
    start = Activity("start", 10)
    b     = Activity("b",     20)
    c     = Activity("c",      5)
    d     = Activity("d",     10)
    f     = Activity("f",     15)
    g     = Activity("g",      5)
    h     = Activity("h",     15)
    end   = Activity("end",   20)

    graph = {start: [f,b,h],
             b    : [c],
             c    : [g,d],
             d    : [end],
             f    : [g],
             g    : [end],
             h    : [end],
             end  : []}
  \end{lstlisting}

The CPM model return two parameters. The first one is the critical path time value (i.e., a float),
while the second one is the actual critical path which is represented as a sequence of activity ID
that are part of the critial path (i.e., a string) separated by an underscore.
The RAVEN ID for the critical path time value is specified in the \xmlNode{CPtime}.
The RAVEN ID for the critical path is specified in the \xmlNode{CPid}.


